\documentclass[12pt]{article}
\usepackage[portuguese]{babel}
\usepackage{natbib}
\usepackage{url}
\renewcommand{\baselinestretch}{1.15} 
\usepackage[utf8x]{inputenc}
\usepackage{amsmath}
\usepackage[table,xcdraw]{xcolor}
\usepackage{graphicx}
\usepackage{parskip}
\usepackage{float}
\usepackage{array}
\usepackage{indentfirst}
\usepackage{fancyhdr}
\usepackage{vmargin}
\usepackage{multirow}
\usepackage{amsfonts} 
\usepackage{titlesec}
    \usepackage{minted}

\title{\vspace{-4em} Auto-Revisão do Teste}
\author{ Henrique Gabriel dos Santos Neto - PG47238 }
\date{8 de abril de 2022}

\makeatletter
\AtBeginEnvironment{minted}{\dontdofcolorbox}
\def\dontdofcolorbox{\renewcommand\fcolorbox[4][]{##4}}
\makeatother

\begin{document}

\maketitle

\section*{Exercício 1}

O exercício contem dois erros que afectam somente a sintaxe do que está escrito, que são resultantes de eu ter a escrito a implementação com a nomenclatura do pseudo-código. Para ambas as variáveis usei os nomes do pseudo-código ($a$ e $b$) em vez do nomes dos argumentos da função ($u$ e $v$). Estou a fazer a chamada recursiva com o nome da função do pseudo-código $gcd$ em vez do nome com que a função está a ser declarada $euclid$. Esta é também por esta razão que na pós-condição a função $gcd$ importada do modulo $Gcd$ é especificada como $"Gcd.gcd"$. Tirando as trocas entre $a-u$ e $b-v$, a pré-condição, a pós-condição e o variante estão correctos e para tal o \textit{why3} conseguiu prová-los.

O ficheiro do exercício para o \textit{why3} já com as correcções da sintaxe é:

\begin{minted}[
frame=lines,
framesep=1.5mm,
baselinestretch=1.2,
fontsize=\footnotesize,
linenos
]{ocaml}
module EuclideanAlgorithm

    use import int.Int
    use import int.EuclideanDivision
    use import ref.Refint
    use import number.Gcd

    let rec euclid (u v : int) : int
        requires {u > 0 /\ v > 0 } 
        ensures  {result = gcd u v}
        variant  {u + v}
    = if u = v then u else (
         if u > v then euclid (u-v) v else euclid u (v-u)
         )
end
\end{minted}


\section*{Exercício 2}

Como agora posso consultar o conteúdo da $stdlib$, posso confirmar que existe de facto existe a função $num\_occ$ para as listas no modulo $list.NumOcc$, por isso não era necessário defini-la no teste. As funções das aulas que foram usadas do $BTree$ e que originaram nas aulas estão presentes no anexo \ref{bter}. Na resolução do teste existiam erros ao importar os modules, por exemplo o module para o predicado $sorted$ é $SortedInt$ e não $Sorted$, a defenição de $++$ está no modulo $Append$ e não é contida no modulo base $List$, etc... 
\begin{minted}[
frame=lines,
framesep=1.5mm,
baselinestretch=1.2,
fontsize=\footnotesize,
linenos
]{ocaml}
module Exercicio2

    use int.Int
    use list.List
    use list.NumOcc
    use list.SortedInt
    use list.Append

    type tree 'a = Empty | Node (tree 'a) 'a (tree 'a)
    
    (* ... Anexo A ... *)
    
    let rec function tree_to_list (t:tree int) : list int
        requires {sortedBT t}
        ensures  {sorted result}
        ensures  {forall x : int. num_occ x t = NumOcc.num_occ x result}
        variant  { t }
    = match t with
        | Empty -> Nil
        | Node t1 x t2 -> (tree_to_list t1) ++ (Cons x (tree_to_list t2))
      end

end
\end{minted}

\section*{Exercicio 4}

Ao contrário do que eu escrevi no teste não é necessário especificar o variante de ciclo para este caso.
Adicionalmente, nos invariantes referi-me aos valores de $i$ como $!i$ , porém isto não é correto, e estes são antes acedidos normalmente com $i$.

Por fim para provar as especificações é necessário especificar um invariante de ciclo muito semelhante à ultima pós-condição que não foi especificado durante o teste. Este indica que para qualquer inteiro $x$, o nº de ocorrências de x no intervalo de $0$ a $i$ é menor ou igual ao nº de ocorrências de $r$ no intervalo de $0$ a $i$, ou seja \mintinline{ocaml}{forall x : int. (numof a x 0 i)  <= (numof a r 0 i)}

Com isto o $why3$ consegue então provar a especificação, sendo esta descrita por:


\begin{minted}[
frame=lines,
framesep=1.5mm,
baselinestretch=1.2,
fontsize=\footnotesize,
linenos
]{ocaml}
module Exercicio4 

    use int.Int
    use ref.Refint
    use array.Array
    use array.NumOfEq
    use array.IntArraySorted

    let most_frequent (a : array int) : int
        requires {length a > 0} (* senão r pode ser invalido no inicio*)
        requires {sorted a}             
        ensures  { (old a) = a}
        ensures  { forall x : int. 
                    (numof a x 0 (length a))  <= (numof a result 0 (length a)) }
    = let ref r = a[0] in 
      let ref c = 1 in
      let ref m = 1 in 
      for i = 1 to length a - 1 do
        invariant {m = numof a r 0 i}
        invariant {c = numof a (a[i-1]) 0 i}
        invariant {forall x : int. (numof a x 0 i)  <= (numof a r 0 i)}
        if a[i] = a[i-1] then begin
            incr c;
            if c > m then begin m <- c ; r <- a[i] end
        end else
            c <- 1
        done;
        r

end
\end{minted}




\newpage
\appendix

\section{Funções e predicados sobre $BTrees$}\label{bter}
\begin{minted}[
baselinestretch=1.2,
fontsize=\footnotesize,
linenos
]{ocaml}

    let rec predicate memt  (t : tree int) (x : int)
        = match t with
            | Empty -> false
            | Node t1 k t2 -> if x = k then true else memt t2 x || memt t1 x
        end

    let rec predicate leq_tree (x : int) (t : tree int)
        ensures { result <-> forall k : int. not (memt t k) \/ k <= x }
        = match t with
            | Empty -> true
            | Node t1 k t2 -> k <= x && leq_tree x t1 && leq_tree x t2
        end

    let rec predicate geq_tree (x : int) (t : tree int)
        ensures { result <-> forall k : int. not (memt t k) \/ k >= x }
        = match t with
            | Empty -> true
            | Node t1 k t2 -> k >= x && geq_tree x t1 && geq_tree x t2
        end

    let rec function num_occ (x : int) (t : tree int) : int
        = match t with
            | Empty -> 0
            | Node t1 k t2 -> if k = x then  1 + (num_occ x t2) + (num_occ x t1) 
                                else (num_occ x t2) + (num_occ x t1)
        end

    inductive sortedBT (t : tree int) = 
        | Sorted_Empty : sortedBT Empty
        | Sorted_Node : forall x :int , t1 t2 : tree int. geq_tree x t1 /\ 
                                        leq_tree x t2 /\ sortedBT t1 /\
                                        sortedBT t2 -> sortedBT (Node t1 x t2)


\end{minted}
\end{document}
